\chapter{Conclusion and Future Work}

In our work, we address the challenges faced by the current most famous model of collaborative filtering. While considering popular and regularly watched videos the user based filtering model gives an accurate recommendation. However for fresh and tail content videos, the collaborative model fails horribly. Our model takes into account the topics of individual videos. Hence this problem is tackled. User engagement increases by using our model of user feedback. \par
Topicality based retrieval also outperforms collaborative filtering in terms of avoiding the various biases which affect the retrieval system in the form of noise. The hybrid model expressed in the last section would include the best of both worlds.\par
As the size of the movie database increases it wouldn’t be possible to perform computation for the entire database especially in the case of multiple clients. Hence some optimization techniques like WeakAnd needs to be implemented which would consider only a subset of the dataset. This random sample collected from the dataset should accurately describe the dataset. WeakAnd query optimization scores a subset of the entire document so that it crosses a given threshold. \par
The performance of the model can be further improved by merging it with the collaborative filtering model. However, in this case there would be a need to rerank all the results obtained from both the topic based model and the collaborative model. This reranking function should not be biased towards any one model. Hence the parameters used must either be common for both the models and must have equal weightage in the models or must not overlap with any of the models. The latter one is prefered as it introduces a new metric which might further improve the recall and precision. This hybrid model could outperform both the models.\par
Dynamic updation of the topic weights could make our model an evolutionary one. Given enough time, our model would keep improving and performing better.